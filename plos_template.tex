% Template for PLoS
% Version 1.0 January 2009
%
% To compile to pdf, run:
% latex plos.template
% bibtex plos.template
% latex plos.template
% latex plos.template
% dvipdf plos.template

\documentclass[10pt]{article}

% amsmath package, useful for mathematical formulas
\usepackage{amsmath}
% amssymb package, useful for mathematical symbols
\usepackage{amssymb}

% graphicx package, useful for including eps and pdf graphics
% include graphics with the command \includegraphics
\usepackage{graphicx}

% cite package, to clean up citations in the main text. Do not remove.
\usepackage{cite}

\usepackage{color} 

% Use doublespacing - comment out for single spacing
%\usepackage{setspace} 
%\doublespacing


% Text layout
\topmargin 0.0cm
\oddsidemargin 0.5cm
\evensidemargin 0.5cm
\textwidth 16cm 
\textheight 21cm

% Bold the 'Figure #' in the caption and separate it with a period
% Captions will be left justified
\usepackage[labelfont=bf,labelsep=period,justification=raggedright]{caption}

% Use the PLoS provided bibtex style
\bibliographystyle{plos2009}

% Remove brackets from numbering in List of References
\makeatletter
\renewcommand{\@biblabel}[1]{\quad#1.}
\makeatother


% Leave date blank
\date{}

\pagestyle{myheadings}
%% ** EDIT HERE **

% \usepackage[english]{babel}
\usepackage[utf8]{inputenc}

%% ** EDIT HERE **
%% PLEASE INCLUDE ALL MACROS BELOW

%% END MACROS SECTION

\begin{document}

% Title must be 150 characters or less
\begin{flushleft}
{\Large
\textbf{Title}
}
% Insert Author names, affiliations and corresponding author email.
\\
Matthias Grenié$^{1}$, 
Jean-François Goût$^{2}$, 
Michael Lynch$^{2}$
\\
\bf{1} Départment de Biologie, École Normale Supérieure de Lyon, Lyon, France
\\
\bf{2} Biology Department, Indiana University, IN, United States of America
\\
\end{flushleft}

% Please keep the abstract between 250 and 300 words
\section*{Abstract}

% Please keep the Author Summary between 150 and 200 words
% Use first person. PLoS ONE authors please skip this step. 
% Author Summary not valid for PLoS ONE submissions.  
\section*{Author Summary}

\dropcap{S}tructure of the introduction
\newline
\begin{itemize}
\item Whole Genome Duplications background, major evolutionary force
\item the \textit{Paramecium} project, why \textit{Paramecium} is interesting, the aurelia complex
\item Here, focusing on the computational part, developing pipeline, showed that etc.
\end{itemize}

Since Ohno first hypothesized the influence of Whole-Genome Duplications (WGD), scientists kept showing that a broad number of species experienced WGD at least once: yeast, Drosophila, Vertebrates, Salmonids, Paramecium. WGD are evolutionary event when the genome of a given individual is duplicated, meaning that the whole genome is in two copies, duplicated pairs of genes are \textbf{paralogs}. WGD may be involved in many evolutive radiations as it creates a context of loosen selection. According to the Duplication-Degeneration-Complementation model,

To understand the consequences of WGDs we have been studying the \textit{Paramecium aurelia} complex. As one of the only free-living eukaryotes studied, other than yeasts, Paramecium is a very attractive model. The diversity of the \textit{Paramecium} ciliates is well studied. We focused on four species of \textit{Paramecium}: \textit{P. biaurelia}, \textit{P. sexaurelia}, \textit{P. tetraurelia} and \textit{P. caudatum} as an outgroup (see phylogenetic tree). The three \textit{aurelia} species underwent two rounds of WGDs, WGDX (... years ago) and WGDX (... years ago).

\textbf{Biological questions:} Do gene expression is linked, in \textit{P.}, with specific motifs? How are TFBS affected by WGD? Are they conserved among species, is this linked to expression level? Conserved among each species? Is there a bias of TFBS usage in certain species?

\section*{Methods}

Used TranslatorX, PhyML, BigFoot, MEME.

We set up a pipeline to make our analyse (Fig. ), the code is available at.

\subsection*{Genomes and Annotation}

We used annotation and sequence from our previous analyses for P. and P. etc. (see .) and the additionnal sequence of.

\subsection*{Gene families}

Looking at the phylogenetic tree of the \textit{aurelia} species, two WGDs occured at the root and affected three of our species. We have established some gene families from WGD2 using comparison, each family contains a set of orthologous genes between the four \textit{aurelia} species studied, and eventually, the paralogous gene found in each species ; at maximum the families contain 13 different genes. Those families were established previously in our team. For details in the method see

\subsection*{Upstream sequences extraction}

We considered only families with at least 4 genes. Then, using our genome assembly and annotation of each species, we extracted upstream regions, upstream of the start codon, from 15nt with a cut-off at 250nt of all genes of the family. If the upstream region of a gene overlapped with another gene we discarded the gene, if the upstream region was less than 15nt long we also discarded the gene. Considering discarded genes, we kept only families with at least 4 members in our datasets.

\subsection*{Coding Sequences extraction and alignment}

Phylogenetic footprinting requires a phylogenetic tree to weigh the evolutionary signal of given motifs. A motif conserved between two close species will have less importance than a motif conserved in two distant species. Because we are focusing on the conservation of upstream sequences we chose not to use them to avoid circularity. Instead,corresponding coding sequences (CDS) were extracted and used to model phylogenetic trees for each family. We preferred to have a gene tree over species tree, to avoid eventual inconsistencies because of gene conversion (ref. needed). See Challenges section for explanations on the use of gene tree over species tree.

CDSs in each family were aligned using TranslatorX (ref. needed) a protein-guided alignment software. The Maximum Likelihood (ML) tree was then computed using PhyML (ref. needed) default parameters.

\subsection*{Phylogenetic footprinting}

We used a phylogenetic footprinting software BigFoot (ref. needed) to detect highly conserved motifs in upstream sequences. We used 10000 burn-in cycles and 20000 cycles with a sampling rate of 1000 for the Hidden Monte-Carlo Markov Chain (HMCMC) process. BigFoot aligns the given sequences with gaps and tries to identify conserved and non-conserved regions ; it models the evolution of those regions along the phylogenetic tree assuming conserved regions evolve more slowly than non-conserved ones. At the end of the analysis BigFoot outputs an alignment of sequences used to identify slow and fast evolving regions as well as, for each nucleotide in the alignment, the posterior probability of the alignment, higher values show higher confidence in the alignment, and the phylogenetic footprinting result, higher values indicating higher posterior probability of purifying selection.

Using a phylogenetic footprinting program means we have to use a phylogenetic tree and depending on the phylogenetic tree we are using, the evolutionary signal used in the footprinting is not identical.

The species tree (see figure.) gives us the relation between all considered, accounting for the various splits between species along with WGDs. The problem is that, not all gene families follow this tree. Because of the successive round of WGDs there are several fates possible for pair of duplicate after the first WGD. Some of these genes may cluster together in the same leaves, if the pairs diverge between species ; another possible outcome is the subfunctionnalization of each gene before the second WGD, meaning before the speciation of the \textit{aurelia} complex, thus genes from different species would cluster together in a phylogenetic tree ; or even a combination of the previous outcome and gene conversion, leading paralogs to recombine in a copy-paste way, changing dramatically the gene tree.

Using these scores we detected motifs of at least 6 nucleotides long, alignment score over 0.8 and phylogenetic score over 0.9. Because of the known biological nature of Transcription Factor Binding Sites, we allowed for a "gap" in motifs of 2 nucleotide, so that the scores could drop under the thresholds in those gaps.

BigFoot does not output directly identified motifs, instead it produces two files with an alignment of the sequences and associated phylogenetic and alignment scores, as explained above.

Transcription factor binding sites are known to be generally conserved but degenerate on certain positions. For example, ... showed that this motif was conserved ....NN... with two highly variable positions (denoted by "N", meaning "A", "T", "C" or "G" using IUPAC notation). Thus, to seek biologically relevant motifs, we had to take into account that in the middle of motifs, the phylogenetic score could drop on several positions, before rising again.

To answer this problem we use a sliding window method of 8 nucleotides in our analysis: for each family, we looked at the scores of 8 nucleotides at the time and slide along the sequences. If the window contained at least 6 bases with scores above our thresholds, we would retain this motif. Then, from this particular region we would try to extend the sequence by adding adjacent nucleotides with good scores.

\subsection*{Comparison with MEME}

To check our predictions and assess the conservation of found motifs, we compared motifs prediction with those of MEME, a widely used \textit{ab initio} motif finding tools. It searches for statistically significant motifs, with a gap-less, local multi-alignment system.

MEME was shown to have a very high False Positive Rate of the discovery (Ref. needed). That is why many studies combine multiple motif detection tools (Ref. needed). In our case, MEME is of particular interest as it obtains motifs using a totally different method from BigFoot.

To assess the relevance of motifs found from BigFoot's outputs, we compared them to a well-known motif finding program: MEME. For each family we identified overlapping motifs between MEME and BigFoot. We computed an overlapping index as follows: 
\begin{equation}
0 \leq \frac{nucleotides in common}{size of the smallest motif} \leq 1
\end{equation}
if this index was over $0.9$ we would then considered the motif as relevant.

\subsection*{Motif classification and data analysis}

All the analyses were produced using R, scatter plots and graphs were produced using the R package ggplot2.



\section*{Results}
\label{sec:Results}

From our orthologs and paralogs gene familes we had 5751 families, only 5008 gene families had at least 4 genes. We ran our pipeline on these 5008 families, to detect conserved motifs among upstream sequences of \textit{Paramecium} genomes. The pipeline was set to detect motifs, in upstream regions ranging between 15nt and 250nt (see~\nameref{sec:Methods}) , of at least 6 nucleotides with an eventual gap in conservation (see~\autoref{fig:DegenerateMotif}) for more biological relevance.

Using BigFoot we identified 811 different motifs in 608 families. Comparing these motifs with MEME ones for partial match, only 117 unique motifs were retained. For each of these, we extracted the name of all genes containing the motif, in each species. To focus only on conserved motifs, we conserved only motifs matching in all the four species. Only 10 motifs matched these conditions. We then computed the ratio of genes containing the motif over the total number of genes (results are summed up in~\autoref{tab:Motifs}).

\section*{Perspectives}

After identifying motifs should relate presence/absence of motifs to duplicated genes fate.

Motifs detection should take phylogeny into account for comparative analysis. Not the same value.

Need to improve pipeline for degenerate motifs, for the moment, extract only exact motif in the genome. (Ref. needed)

Need to measure diversity among detected motifs -> clustering tools and suppress rendundancy

Implement other motif finding tools to validate results.


% Do NOT remove this, even if you are not including acknowledgments
\section*{Acknowledgments}

I would to thank my advisor Jean-François Goût for his patient and constant support, Michael Lynch for having me in his lab. More broadly, I would like to thank the whole Lynch lab team, for great scientific and non-scientific discussions.

\section*{References}
% The bibtex filename
\bibliography{template}

\section*{Figure Legends}

\begin{figure}[!ht]
\begin{center}
\includegraphics{Figures/TreeOfEuk.png}
\end{center}
\caption{
{\bf Phylogenetic tree of eukaryotes phylas.} The \textit{Ciliates} group, underlined in red in the figure, inside the Alveolates among the Chromoalveloates, it contains the \textit{Paramecium} genus. While the Animals (Metazoa) belong to the Opisthokonts group. From (Ref. needed)
} 
\label{TreeOfEuk}
\end{figure}


\begin{figure}[!ht]
\begin{center}
\includegraphics{Figures/Pipeline.png}
\end{center}
\caption{
{\bf Flow chart of the whole pipeline.}  From the genome assembly and annotation of the four species, we used orthology and paralogy groups from (Ref. needed). For each of these families we extracted the coding sequences (CDS) as well as upstream regions from the start codon. We built a maximum likelihood phylogenetic tree using PhyML on pre-aligned CDS with TranslatorX. On the one hand we computed the first fifth motifs of size of at least 4 nt using MEME on upstream regions, while on the other hand using both the tree and the upstream sequences were used to detect motifs with a phylogenetic footprinting approach using BigFoot. We then retained only conserved motifs between MEME and BigFoot.
}
\label{Pipeline}
\end{figure}

\begin{figure}[!ht]
\begin{center}
\includegraphics{Figures/DuplicationPatterns.png}
\end{center}
\caption{
{\bf One possible gene tree.} An example of a family gene tree. A single \textit{caudatum} gene and four genes for each \textit{aurelia} species. \textit{Legend:} blue circle, intermediary WGD; orange circle, more recent WGD; as dated by (Ref. needed). Adapted from (Ref. needed)
}
\label{DuplicationTree}
\end{figure}


\section*{Tables}
%\begin{table}[!ht]
%\caption{
%\bf{Table title}}
%\begin{tabular}{|c|c|c|}
%table information
%\end{tabular}
%\begin{flushleft}Table caption
%\end{flushleft}
%\label{tab:label}
% \end{table}

\end{document}

