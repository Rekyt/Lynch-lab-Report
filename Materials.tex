\section*{Methods}

We set up a pipeline to make our analyse (Fig.~\ref{fig:Pipeline}), the code is available at.

\subsection*{Genomes and Annotation}

We used annotation and sequence from our previous analyses for \textit{P. caudatum} \& the species from the \textit{aurelia} complex. (see~\citet{mcgrath_insights_2014})

\subsection*{Gene families}

Looking at the phylogenetic tree of the \textit{aurelia} species, two WGDs occured at the root and affected three of our species. We have established some gene families from WGD2 using comparison, each family contains a set of orthologous genes between the four \textit{aurelia} species studied, and eventually, the paralogous gene found in each species ; at maximum the families contain 13 different genes. Those families were established previously in our team. For details in the method see (~\citet{mcgrath_insights_2014} \& McGrath \textit{et al.} in press)

\subsection*{Upstream sequences extraction}

For each gene, we extracted 250nt upstream of the start codon. If the previous gene was less than 250nt away, we reduced the extracted region so that it includes only inter-genic region. If the region was less than 15nt long, we removed the gene from the family. Sequences were extracted so that the first nucleotide of each one was the nearest from the start codon.

\subsection*{Coding Sequences extraction and alignment}

Phylogenetic footprinting requires a phylogenetic tree to weigh the evolutionary signal of given motifs. A motif conserved between two close species will have less importance than a motif conserved in two distant species. Because we are focusing on the conservation of upstream sequences we chose not to use them to avoid circularity. Instead,corresponding coding sequences (CDS) were extracted and used to model phylogenetic trees for each family. We preferred to have a gene tree over species tree, to avoid eventual inconsistencies because of gene conversion (ref. needed). See Challenges section for explanations on the use of gene tree over species tree.

CDSs in each family were aligned using TranslatorX (ref. needed) a protein-guided alignment software. The Maximum Likelihood (ML) tree was then computed using PhyML (ref. needed) the HKY85 model.

\subsection*{Phylogenetic footprinting}

We used a phylogenetic footprinting software BigFoot (ref. needed) to detect highly conserved motifs in upstream sequences. We used 10000 burn-in cycles and 20000 cycles with a sampling rate of 1000 for the Hidden Monte-Carlo Markov Chain (HMCMC) process. BigFoot aligns the given sequences with gaps and tries to identify conserved and non-conserved regions ; it models the evolution of those regions along the phylogenetic tree assuming conserved regions evolve more slowly than non-conserved ones. At the end of the analysis BigFoot outputs an alignment of sequences used to identify slow and fast evolving regions as well as, for each nucleotide in the alignment, the posterior probability of the alignment, higher values show higher confidence in the alignment, and the phylogenetic footprinting result, higher values indicating higher posterior probability of purifying selection.

Using a phylogenetic footprinting program means we have to use a phylogenetic tree and depending on the phylogenetic tree we are using, the evolutionary signal used in the footprinting is not identical.

The species tree (see figure.) gives us the relation between all considered, accounting for the various splits between species along with WGDs. The problem is that, not all gene families follow this tree. Because of the successive round of WGDs there are several fates possible for pair of duplicate after the first WGD. Some of these genes may cluster together in the same leaves, if the pairs diverge between species ; another possible outcome is the subfunctionnalization of each gene before the second WGD, meaning before the speciation of the \textit{aurelia} complex, thus genes from different species would cluster together in a phylogenetic tree ; or even a combination of the previous outcome and gene conversion, leading paralogs to recombine in a copy-paste way, changing dramatically the gene tree.

Using these scores we detected motifs of at least 6 nucleotides long, alignment score over 0.8 and phylogenetic score over 0.9. Because of the known biological nature of Transcription Factor Binding Sites, we allowed for a "gap" in motifs of 2 nucleotide, so that the scores could drop under the thresholds in those gaps.

BigFoot does not output directly identified motifs, instead it produces two files with an alignment of the sequences and associated phylogenetic and alignment scores, as explained above.

Transcription factor binding sites are known to be generally conserved but degenerate on certain positions. For example, ... showed that this motif was conserved ....NN... with two highly variable positions (denoted by "N", meaning "A", "T", "C" or "G" using IUPAC notation). Thus, to seek biologically relevant motifs, we had to take into account that in the middle of motifs, the phylogenetic score could drop on several positions, before rising again.

To answer this problem we use a sliding window method of 8 nucleotides in our analysis: for each family, we looked at the scores of 8 nucleotides at the time and slide along the sequences. If the window contained at least 6 bases with scores above our thresholds, we would retain this motif. Then, from this particular region we would try to extend the sequence by adding adjacent nucleotides with good scores.

\subsection*{Comparison with MEME}

To check our predictions and assess the conservation of found motifs, we compared motifs prediction with those of MEME, a widely used \textit{ab initio} motif finding tools. It searches for statistically significant motifs, with a gap-less, local multi-alignment system.

MEME was shown to have a very high False Positive Rate of the discovery (Ref. needed). That is why many studies combine multiple motif detection tools (Ref. needed). In our case, MEME is of particular interest as it obtains motifs using a totally different method from BigFoot.

To assess the relevance of motifs found from BigFoot's outputs, we compared them to a well-known motif finding program: MEME. For each family we identified overlapping motifs between MEME and BigFoot. We computed an overlapping index as follows: 
\begin{equation}
0 \leq \frac{\text{nucleotides in common}}{\text{size of the smallest motif}} \leq 1
\end{equation}
if this index was over $0.9$ we would then considered the motif as relevant.

\subsection*{Motif classification and data analysis}

All the analyses were produced using R, scatter plots and graphs were produced using the R package ggplot2.

