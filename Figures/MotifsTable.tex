% latex table generated in R 3.1.0 by xtable 1.7-3 package
% Sun Jun  1 01:19:19 2014
\begin{table}[!ht]

%\noindent\makebox[\textwidth]{%
\resizebox{0.7\textwidth}{!}{\begin{minipage}{\textwidth}
\rowcolors{1}{light-gray}{white}
\begin{tabular}{lccccccccccccc}
  \hline
 \hiderowcolors Motif & Ratio bi. & Ratio ca. & Ratio sex. & Ratio tet. & Type bi. & Type ca. & Type sex. & Type tet.\\ 
  \hline
 \showrowcolors AAAAAT & 44.74 (17555) & 25.60 (4739) & 51.15 (17872) & 38.61 (15833) & \textit{NS} & \textit{NS} & \textit{NS} & \textbf{H}  \\ 
  TAAATCT & 5.97 (2341) & 2.54 (471) & 6.25 (2184) & 5.18 (2124) & \textbf{H}  & \textbf{H}  & \textit{NS} & \textit{NS} \\ 
  TAAATT & 52.36 (20549) & 31.23 (5780) & 56.81 (19849) & 49.60 (20340) & \textbf{H}  & \textit{NS} & \textit{NS} & \textbf{H}  \\ 
  TATTTA & 52.00 (20406) & 31.26 (5786) & 54.76 (19131) & 47.02 (19283) & \textbf{H}  & \textit{NS} & \textit{NS} & \textbf{H}  \\ 
  TTAAAT & 51.47 (20199) & 30.36 (5619) & 55.55 (19407) & 48.11 (19728) & \textbf{H}  & \textbf{H}  & \textit{NS} & \textit{NS} \\ 
  TTAAATT & 27.34 (10727) & 14.86 (2750) & 30.57 (10680) & 25.59 (10493) & \textbf{H}  & \textit{NS} & \textbf{H}  & \textbf{H}  \\ 
  TTAATATT & 13.89 (5452) & 7.68 (1421) & 14.18 (4956) & 12.26 (5026) & \textbf{H}  & \textit{NS} & \textit{NS} & \textit{NS} \\ 
  TTAATT & 58.09 (22797) & 40.53 (7501) & 60.91 (21282) & 55.34 (22693) & \textbf{H}  & \textit{NS} & \textit{NS} & \textit{NS} \\ 
  TTAATTA & 32.62 (12802) & 23.29 (4311) & 34.27 (11975) & 30.30 (12427) & \textbf{H}  & \textbf{H}  & \textbf{H}  & \textbf{H}  \\ 
  TTTATT & 54.12 (21238) & 32.14 (5949) & 58.26 (20357) & 49.61 (20344) & \textbf{H}  & \textit{NS} & \textit{NS} & \textbf{H}  \\ 
  \hline
\end{tabular}
\end{minipage}}

\caption{
{\bf Common motifs in all species.} bi: \textit{P. biaurelia}, ca: \textit{P. caudatum}, sex: \textit{P. sexaurelia}, tet: \textit{P. tetraurelia}. Match: number of genes containing the motif. Ratio: Proportion of genes containing the motif over all genes. Type: is the motif associated with a higher expression level? H: associated with higher expression, NS: non-significant.}
\label{tab:Motifs}


\end{table}