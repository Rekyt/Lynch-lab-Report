\section*{Abstract}

Whole-Genome Duplication (WGD) are widespread among the tree of life, in plants, animals or even Ciliates. Duplicate genes have different rate of retention after a WGD, some of them are fully conserved while other are lost. \textit{Paramecium} is a great system to study WGD, especially because of the numerous genomes available. In addition, the \textit{aurelia} complex is a cryptic species complex, whom species experienced two rounds of WGDs. Recently, retention rate was correlated to duplicate retention in \textit{Paramecium}, the higher expressed a gene was, the greater chances of retention he had. Thus, trying to understand the links between cis-regulatory sequences and gene expression would let us eventually understand better gene retention. Here we used predefined orthology and paralogy gene families \citep{mcgrath_insights_2014} to identify conserved motifs in upstream regions of genes in \textit{Paramecium caudatum}, \textit{P. biaurelia}, \textit{P. sexaurelia} and \textit{P. biaurelia}. We developed a phylogenetic approach with BigFoot \citep{satija_bigfoot:_2009}, complemented by analyses with MEME \citep{bailey_meme:_2006}. We identified more than 117 motifs conserved among all species. We found that 10 of them were present in the 4 species at various rates. \textit{P. caudatum} seemed to have a smaller of ratio of gene containing conserved motifs. We still have to improve our analysis pipeline to identify better candidates and suppress redundancy in found motifs.