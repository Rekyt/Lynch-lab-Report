\section*{Results}
\label{sec:Results}

From our orthologs and paralogs gene familes we had 5751 families, only 5008 gene families had at least 4 genes. We ran our pipeline on these 5008 families, to detect conserved motifs among upstream sequences of \textit{Paramecium} genomes. The pipeline was set to detect motifs, in upstream regions ranging between 15nt and 250nt (see~\nameref{sec:Methods}), of at least 6 nucleotides with an eventual gap in conservation (see~\autoref{fig:DegenerateMotif}) for more biological relevance.

Using BigFoot we identified 811 different motifs in 608 families. For each of these, we extracted the name of all genes containing the motif, in each species. Only 10 motifs were present in the four species. We then computed the ratio of genes containing the motif over the total number of genes (results are summed up in~\autoref{tab:Motifs}).

One might note that all retained motifs seem to be modified TATA box, also, the small number of match may be explained by too stringent conditions. Still, if we look at the proportions of matching genes in each species, we can see different patterns (see~\autoref{fig:PropFigure}). First, looking by species (\autoref{fig:PropFigure}A), \textit{P. caudatum} seems to have a lower ratio than the other species. Indeed, we proceeded to a GLM as explained in~\nameref{sec:Methods} section, and found a very significant difference between the proportions (p-value \textless~2e-16). Two hypotheses can explain this pattern: either \textit{P. caudatum} lost the gene with these motifs, or the three \textit{aurelia} species retained more genes with them.

Looking at all the different motifs (\autoref{fig:PropFigure}C), we can see that between two and four motifs have a lower of genes containing them: \texttt{TAAATCT} has a mean ratio around 5\% of the genes, \texttt{TTAATATT} around 12\% and \texttt{TTAAATT} \& \texttt{TTAATTA} around 30\%, while other motifs have mean proportion around 40\%. This may be due to different functions done by motifs, some motifs may be very specific to a given category of genes, while others are more widely used.

Indeed \autoref{fig:PropFigure}B underlines the differences between motifs, to compare the proportions between species, we computed a GLM for each motifs, then used an ANOVA on this GLM. All the tests were highly significant (pvalue \textless 2e-16) showing that in each motifs, the ratios between species are significantly different. Because we observed overall lower ratios in \textit{P. caudatum}, we redid the analysis without it and all the comparisons were still significant. Thus, motif do not seem to be spread in the same way in each species.

We also tried to classify each motif as increasing or decreasing the overall expression of genes. We compared the expression distribution of genes with and without a motif using a Wilcoxon test, if it was significant, than we would classify the motif as ''increasing the expression'' or ''decreasing the expression'' according on the difference between means (results in \autoref{tab:Motifs}). All the found motifs seem to be associated with an increase of expression for the genes possessing them in at least a species and 9 out of 10 at least in two species. \texttt{TTAAATT} and \texttt{TTAATTA} are associated with high expression levels in, respectively, 3 and 4 species, they may be good candidates to study the evolutionary fate of genes.

For the 117 unique found motifs, we also correlated the average expression levels with the distance from the start codon. For the 10 retained motifs we found a very significant (pvalue \textless~5e-5 for all motifs) positive correlation with distances (data not shown). The farer the motif from the start codon the higher the expression of the genes containing it would be.