\section*{Introduction}

Structure of the introduction
\newline
\begin{itemize}
\item Whole Genome Duplications background, major evolutionary force
\item the \textit{Paramecium} project, why \textit{Paramecium} is interesting, the aurelia complex
\item Here, focusing on the computational part, developing pipeline, showed that etc.
\end{itemize}

Since Ohno first hypothesized the influence of Whole-Genome Duplications (WGD), scientists kept showing that a broad number of organisms experienced several rounds of WGD: yeast, insects, Angiosperma, Vertebrates, Salmonids, and many others. WGD are evolutionary event when the genome of a given individual is duplicated, meaning that the whole genome is in two copies, duplicated pairs of genes are \textbf{paralogs}. WGD can also occur when two closely related species hybridize and form a fertile descendant.

WGDs may be involved in many evolutive radiations as they provide the raw material to explore new evolutionary landscapes. A recent study on the horshoe crab genome underlined that WGDs may be a more common phenomenon than stated. They showed that at least a WGD occurred in this conserved lineage, thus WGDs may not be evolutionary drivers.

Still, it seems that WGDs are widespread along the tree of life and since Ohno numerous models have been developed to explain the retention rate we observe between duplicate genes (reviewed in ). Gene Balance Hypothesis is for example a well explored hypothesis in the litterature, according to this hypothesis dosage-sensitive gene are more retained because of stœchiometry problem, it would thus explain the over retention of transcription factors and multicomplexes proteins.

To understand the consequences of WGDs we have been studying the \textit{Paramecium aurelia} complex. \textit{Paramecium} are a Ciliates group. (See position on phylogenetic tree) As one of the only free-living eukaryotes studied, other than yeasts, Paramecium is a very attractive model. The diversity of the \textit{Paramecium} ciliates is well studied. We focused on four species of \textit{Paramecium}: \textit{P. biaurelia}, \textit{P. sexaurelia}, \textit{P. tetraurelia} and \textit{P. caudatum} as an outgroup (see phylogenetic tree). The three \textit{aurelia} species underwent two rounds of WGDs, WGDX (... years ago) and WGDX (... years ago).

We have shown previous the tight link between gene expression and duplicate retention among \textit{P. caudatum} and \textit{P. tetraurelia}. Highly expressed genes are more retained than low expressed genes; the regulation of gene expression thus impacts gene retention. \textit{Cis}-regulatory sequences are upstream sequences influencing downstream gene expression.

Transcription Factor Binding Sites, regulatory sequences, regulatory network?, conserved sequences, studied for a long time. Regulatory sequences <-> WGD?

Objective: detect conserved TFBS among various our species -> to identify candidates for further studies.

\textbf{Biological questions:} Do gene expression is linked, in \textit{P.}, with specific motifs? How are TFBS affected by WGD? Are they conserved among species, is this linked to expression level? Conserved among each species? Is there a bias of TFBS usage in certain species?