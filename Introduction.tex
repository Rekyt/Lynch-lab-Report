\section*{Introduction}

Since Ohno first hypothesized the influence of Whole-Genome Duplications (WGD)~\citep{ohno_enormous_1970}, scientists kept showing that a broad number of organisms experienced several rounds of WGD: \textit{Saccharomyces cerevisiae}~\citep{fares_roles_2013}, Angiosperms~\citep{arrigo_rarely_2012}, Vertebrates~\citep{dehal_two_2005}, Salmonids~\citep{alexandrou_genome_2013}, and many others. WGD are evolutionary event when the genome of a given individual is duplicated, meaning that the whole genome is in two copies, duplicated pairs of genes are called \textit{paralogs}. WGD can also occur after two closely related species hybridize to avoid hybrid incompatibility issues.

WGDs may be involved in many evolutionary radiations~\citep{alexandrou_genome_2013} as they provide the raw material to explore new evolutionary landscapes. A recent study on the horshoe crab genome underlined that WGDs may be a more common phenomenon than stated. They showed that at least a WGD occurred in this conserved lineage, thus WGDs may not be evolutionary drivers.

Still, it seems that WGDs are widespread along the tree of life and since Ohno numerous models have been developed to explain the retention rate we observe between duplicate genes (reviewed in~\citet{innan_evolution_2010}). Gene Balance Hypothesis is for example a well explored hypothesis in the literature, according to this hypothesis dosage-sensitive gene are more retained because of stoichiometry issues, it would thus explain the over retention of transcription factors and multi-complexes proteins.

To understand the consequences of WGDs we have been studying various \textit{Paramecium} species~\citep{beisson_paramecium_2010}. \textit{Paramecium} are a Ciliates group (See Fig.~\ref{fig:TreeOfEuk}). Aury \textit{et al.} showed that at least three round of WGDs occurred in the \textit{Paramecium} genus~\citep{aury_global_2006}, two of which occurred in the \textit{aurelia} complex (see Fig.~\ref{fig:DuplicationTree}). This cryptic species complex of 13 reproductively isolated species is a great model to study the fate of duplicate genes~(\citet{catania_genetic_2009},~\citet{mcgrath_insights_2014}).

Recently, several studies unraveled the link between gene expression in gene retention after WGD in \textit{Paramecium} (~\citet{gout_relationship_2010}, \citet{arnaiz_gene_2010}). Genes that are highly expressed are more likely to be retained than genes with a low expression. The COSTEX model proposed by the authors states that gene evolution is more constrained as gene expression is high. It explains well why highly expressed genes are more retained and why they evolve more slowly thant other genes.

Having established this link between gene retention and expression level of the gene, it is normal to try to get better understanding of \textit{Paramecium} regulatory sequences. The compactness of \textit{Paramecium} genomes makes the study of regulatory elements easier than in other eukaryotic species~\citep{mcgrath_insights_2014}. \textit{Paramecium} genomes have on average 250nt long inter-genic regions close to those of \textit{Saccharomyces cerevisiae}~\citep{chen_minimal_2011}. These short inter-genic regions are known, in \textit{S. cerevisiae}, to regulate genes directly downstream of them. Transcription regulator is thus far easier to study as for each gene its \textit{cis}-regulatory sequence is directly upstream of it and not hundreds of thousand bases away as it can be the case in mammalian genomes. Because \textit{Paramecium} also have very short inter-genic regions, it is reasonable to assume that the promoters share the same mechanisms.

Various strategies to study regulatory elements evolution have been developed~\citep{wittkopp_cis-regulatory_2012}; the idea being that regulatory elements can be conserved along evolution, some tools can detect these more conserved regions and thus output putative motifs. Phylogenetic footprinting is one of the approaches developed when trying to detect motifs among several species~\citep{zhang_mice_2003}, the idea being that regulatory motifs tend to be conserved by purifying selection~\citep{nelson_conserved_2013}, while non-functional elements will accumulate mutation along evolution. The motifs are than weighted by the phylogenetical relationships of the given sequences, \textit{id est}, if a motif is found in two closely related species it will have less weight than if it is detected between two distant species.

In this paper we develop a phylogenetic footprinting workflow to study the \textit{cis}-regulatory elements among three \textit{aurelia} species: \textit{P. biaurelia}, \textit{P. sexaurelia}, \textit{P. tetraurelia} and a more basal species \textit{P. caudatum}. The three \textit{aurelia} experienced two specific rounds of WGD compared to \textit{P. caudatum}, thus when studying orthology and paralogy groups between genes, they can contain up to 13 different genes: 1 from \textit{P. caudatum} and 4 from each \textit{aurelia} species. Using those groups we use our workflow to identify conserved motifs in the upstream regions in each group.

