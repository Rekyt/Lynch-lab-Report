developped a whole pipeline (show simple pipeline graph)
\begin{itemize}
\item Families upstream sequences extraction
\item CDSs extraction and alignment
\item CDSs phylogenetic tree
\item BigFoot identification (explanation of phylogenetic score and alignment score)
\item MEME research
\item Comparison MEME and BigFoot
\item Identification of given motif in species genome
\item Correlation between motifs and expression levels
\end{itemize}

Used TranslatorX, PhyML, BigFoot, MEME.

\subsection{Genomes and Annotation}

We used annotation and sequence from our previous analysis for P. and P. etc. (see .) and the additionnal sequence of.

\subsection{Gene families}

Looking at the phylogenetic tree of the \textit{aurelia} species, two WGDs occured at the root and affected three of our species. We have established some gene families from WGD2 using comparison, each family contains a set of orthologous genes between the four \textit{aurelia} species studied, and eventually, the paralogous gene found in each species. In our dataset we had 5781 families.

\subsection{Upstream sequences extraction}

We considered only families with at least 4 members. Then, using our assembly and annotation of each species genome, we extracted upstream regions from 15nt with a cut-off at 250nt of all genes of the family. If the upstream region of a gene overlapped with another gene we discarded the gene, if the upstream region was less than 15nt long we also discarded the gene. Considering discarded genes, we kept families with at least 4 members in our datasets. Under these conditions, we extracted genes from 5008 families.

