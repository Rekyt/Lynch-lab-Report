\section*{Challenges}

During the pipeline development several methods questions were raised. Leading to choices

\subsection*{Species tree or gene tree?}

Using a phylogenetic footprinting program means we have to use a phylogenetic tree and depending on the phylogenetic tree we are using, the evolutionary signal used in the footprinting is not identical.

The species tree (see figure.) gives us the relation between all considered, accounting for the various splits between species along with WGDs. The problem is that, not all gene families follow this tree. Because of the successive round of WGDs there are several fates possible for pair of duplicate after the first WGD. Some of these genes may cluster together in the same leaves, if the pairs diverge between species ; another possible outcome is the subfunctionnalization of each gene before the second WGD, meaning before the speciation of the \textit{aurelia} complex, thus genes from different species would cluster together in a phylogenetic tree ; or even a combination of the previous outcome and gene conversion, leading paralogs to recombine in a copy-paste way, changing dramatically the gene tree.


\subsection*{Motif detection strategy}


BigFoot does not output directly identified motifs, instead it produces two files with an alignment of the sequences and associated phylogenetic and alignment scores. The phylogenetic scores is computed, etc. The alignment scores, etc.

Transcription factor binding sites are known to be generally conserved but degenerate on certain positions. For example, ... showed that this motif was conserved ....NN... with two highly variable positions (denoted by\"N\", meaning \"A\", \"T\", \"C\" or \"G\" using IUPAC notation). Thus, to seek biologically relevant motifs, we had to take into account that in the middle of motifs, the phylogenetic score could drop on several positions. (Show a phylogenetic score profile?)

To answer this problem we use a sliding window method of 8 nucleotides in our analysis: for each family, we looked at the scores of 8 nucleotides at the time and slide along the sequences. If the window contained at least 6 bases with scores above our thresholds, we would retain this motif. Then, from this particular region we would try to extend the sequence by adding adjacent nucleotides with good scores.



\subsection*{Measure motif relevance}

MEME was shown to have a very high False Positive Rate of the discovery (Ref. needed). That is why many studies combine multiple motif detection tools (Ref. needed). In our case, MEME is of particular interest as it obtains motifs using a totally different method from BigFoot.

To assess the relevance of motifs found from BigFoot's outputs, we compared them to a well-known motif finding program: MEME. For each family we identified overlapping motifs between MEME and BigFoot. We computed an overlapping index as follows: $0 \leq \frac{nucleotides in common}{size of the smallest motif} \leq 1$, if this index was over $0.9$ we would then considered the motif as relevant.