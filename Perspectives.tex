\section*{Future Directions}
\label{sec:Future}

We identified motifs using an automated pipeline (\autoref{fig:Pipeline}), comparing motifs identified using phylogenetic footprinting and statistical enrichment methods. We were able to dectect X motifs, associated with specific expression values and distance. Still, we plan to improve this pipeline in several ways.

After identifying motifs, we scan the entire \textit{Paramecium} genomes to find associated genes, however we only search for perfect matches. Thus, we do not take into account degenerate motifs. Thus we can miss genes that have slightly different motifs. Recently XXX showed that looking exact matches during motif identification may miss most of the signal. Indeed we have increase evidence that real motifs are most of the time degenerate and that Transcription Factor binding mostly depends on the general DNA context.

After identifying motifs, one might want to group the motifs by similarity. We generally think about motifs as families, having specific sequences with some variation. For the moment, in our pipeline, we do not cluster motifs using ''supposed'' family, we consider each exact motifs as unique. However, several clustering methods exist to measure the diversity among motifs. The small sizes of motifs (about 6nt to 15nt) make these methods complicated to develop. Several methods have been implemented, but still, are not adapted for motifs and do not output families of motifs. 

Another way to think about the problem can be to cluster motifs according to features one can measure on them. For example, one can compute several statistical indexes on motifs to compare. Thus we do not need to implement a new algorithm to cluster motifs as other classical methods such as principal components analysis, hierarchal clustering or \textit{k}-means clustering can be applied. Still, these methods rely on the indexes you choose to measure, the motifs you are studying have to have well-defined clusters using those variables. We tried to cluster the studied motifs with this hierarchal clustering method (Fig.).

Several other new methods have been proposed to cluster motifs as they are, considering the sequences and the variability along each nucleotide. Some of them rely on position-specific weight matrices (PSM) which are matrices of the abundances of each base at each position. In a future improvement of the pipeline we cool try to implement such methods to reduce the redundancy of our data.

It has been proven that most motif identification tools have a high false positive discovery rate, thus, we need ways to confirm the relevance of found motifs. One way used by \citealt{liseron-monfils_promzea:_2013} is to increase the number of tools used in the pipeline, in their analysis they used MEME, BioProspector and Weeder and filtered out significant results and then only combined them. They have selected in each set, significant motifs using p-values and then only compare them. Liseron \textit{et al.} showed that by combining these tools they were able to increase the sensitivity by 22\% over the best standalone tool. Here we used MEME and BigFoot but we could include other tools in our workflow.